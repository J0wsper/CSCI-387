\documentclass[11pt, reqno]{amsart}
\usepackage[margin=1in]{geometry}
\geometry{letterpaper}
%\geometry{landscape} % Activate for for rotated page geometry
\usepackage[parfill]{parskip} % Activate to begin paragraphs with an
% empty line rather than an indent
\usepackage{amsfonts, amscd, amssymb, amsthm, amsmath}
\usepackage{mathtools} %xmapsto etc
\usepackage{pdfsync} %leaves makers for tex searching
\usepackage{enumerate}

%%% Color %%%---------------------------------------------------------
\usepackage{color}
\usepackage[dvipsnames]{xcolor}
\definecolor{dred}{HTML}{C30101}
\definecolor{dorange}{rgb}{.9,.3,0}
\definecolor{dgrey}{rgb}{.4,.4,.4}
\definecolor{plumb}{HTML}{8105C1}
\definecolor{pumpkin}{HTML}{E47604}
\definecolor{rose}{HTML}{C10091}
\definecolor{dgreen}{HTML}{25A75B}
\definecolor{dblue}{HTML}{0066FF}
\definecolor{cornflower}{HTML}{3256C3}
\definecolor{viridian}{HTML}{099A97}
\definecolor{alert}{HTML}{3256C3}

\newcommand\plumb[1]{{\color{plumb}#1}}
\newcommand\cornfl[1]{{\color{cornflower}#1}}
\newcommand\dgreen[1]{{\color{dgreen}#1}}
\newcommand\viridian[1]{{\color{viridian}#1}}
\newcommand\dblue[1]{{\color{dblue}#1}}
\newcommand\dred[1]{{\color{dred}#1}}
\newcommand\gray[1]{{\color{black!40}#1}}
\newcommand\black[1]{{\color{black}#1}}
\newcommand\pumpk[1]{{\color{pumpkin}#1}}
\newcommand\rose[1]{{\color{rose}#1}}

\newcommand{\NOTE}[1]{{\color{cornflower}#1}}
\newcommand{\alert}[1]{{\color{alert}#1}}
\newcommand{\Alert}[1]{\emph{\color{alert}#1}}

\usepackage{hyperref} %[pdftex,bookmarks]
\hypersetup{
  colorlinks=true,
  linkcolor=viridian,
  filecolor=viridian,
  citecolor=viridian,
  urlcolor=viridian,
  pdfpagemode=FullScreen,
}

%%% Theorems %%%---------------------------------------------------------
\theoremstyle{plain}
\newtheorem*{thm}{Theorem}
\newtheorem*{lemma}{Lemma}
\newtheorem*{prop}{Proposition}
\newtheorem*{cor}{Corollary}
\theoremstyle{definition}
\newtheorem*{defn}{Definition}
\newtheorem*{remark}{Remark}

%%% Environments %%%---------------------------------------------------------
\newenvironment{pf}{\color{black}\medskip
\paragraph*{\emph{Proof}.}}{\hfill \qedsymbol \medskip }
\newenvironment{ans}{\medskip \color{black}
\paragraph*{\emph{Answer}.}}{\hfill \break $~\!\!$ \dotfill \medskip }
\newenvironment{sketch}{\medskip \paragraph*{\emph{Proof sketch}.}}{ \medskip }
\newenvironment{summary}{\medskip \paragraph*{\emph{Summary}.}}{
\hfill \break \rule{1.5cm}{0.4pt} \medskip }
\newcommand\Ans[1]{$ $\hfill {\color{black}\emph{Answer:} {#1}}}
\newcommand{\Hint}[1]{{\small [\emph{Hint:} {#1}]}}

%%% Pictures %%%---------------------------------------------------------
%%% If you need to draw pictures, tikzpicture is one good option.
% Here are some basic things I always use:
\usepackage{tikz}
\usetikzlibrary{scopes}
\usetikzlibrary{automata}
\usetikzlibrary{positioning}
\usepgflibrary{shapes}
\tikzstyle{V}=[draw, fill =black, circle, inner sep=0pt, minimum size=2pt]
\tikzstyle{bV}=[draw, fill =black, circle, inner sep=0pt, minimum size=4pt]
\tikzstyle{over}=[draw=white,double=black,line width=3pt]
\tikzstyle{C}=[draw, thick, fill =white, circle, inner sep=0pt,
minimum size=6pt]

\newcommand\TikZ[1]{
  \begin{matrix}
    \begin{tikzpicture}#1
    \end{tikzpicture}
\end{matrix}}

\newcounter{r}
\newcommand\Part[1]{
  \setcounter{r}{1}
  \foreach \x in {#1}{
    {\ifnum\value{r}=1
      \draw (0,\value{r}-1)--(\x,\value{r}-1);
    \fi}
    \draw (0,\value{r}) to (\x,\value{r});
    \foreach \y in {0, ..., \x} {\draw (\y,\value{r})--(\y,\value{r}-1);}
    \addtocounter{r}{1}
}}
\def\PartUNIT{.175}
%Self-contained tikz images for \Part above.
\newcommand{\PART}[1]{
  \begin{matrix}
    \begin{tikzpicture}[xscale=\PartUNIT, yscale=-\PartUNIT]
      \Part{#1}
    \end{tikzpicture}
  \end{matrix}
}

\def\FOUR{4}\def\ONE{1}\def\FIVE{5}\def\EIGHT{8}\def\THREE{3}\def\TWO{2}\def\SIX{6}\def\SEVEN{7}

%%% Alphabets %%%---------------------------------------------------------
%%% Some shortcuts for my commonly used special alphabets and characters.
\def\cA{\mathcal{A}}\def\cB{\mathcal{B}}\def\cC{\mathcal{C}}\def\cD{\mathcal{D}}\def\cE{\mathcal{E}}\def\cF{\mathcal{F}}\def\cG{\mathcal{G}}\def\cH{\mathcal{H}}\def\cI{\mathcal{I}}\def\cJ{\mathcal{J}}\def\cK{\mathcal{K}}\def\cL{\mathcal{L}}\def\cM{\mathcal{M}}\def\cN{\mathcal{N}}\def\cO{\mathcal{O}}\def\cP{\mathcal{P}}\def\cQ{\mathcal{Q}}\def\cR{\mathcal{R}}\def\cS{\mathcal{S}}\def\cT{\mathcal{T}}\def\cU{\mathcal{U}}\def\cV{\mathcal{V}}\def\cW{\mathcal{W}}\def\cX{\mathcal{X}}\def\cY{\mathcal{Y}}\def\cZ{\mathcal{Z}}

\def\AA{\mathbb{A}} \def\BB{\mathbb{B}} \def\CC{\mathbb{C}} \def\DD{\mathbb{D}}
\def\EE{\mathbb{E}} \def\FF{\mathbb{F}} \def\GG{\mathbb{G}} \def\HH{\mathbb{H}}
\def\II{\mathbb{I}} \def\JJ{\mathbb{J}} \def\KK{\mathbb{K}} \def\LL{\mathbb{L}}
\def\MM{\mathbb{M}} \def\NN{\mathbb{N}} \def\OO{\mathbb{O}} \def\PP{\mathbb{P}}
\def\QQ{\mathbb{Q}} \def\RR{\mathbb{R}} \def\SS{\mathbb{S}} \def\TT{\mathbb{T}}
\def\UU{\mathbb{U}} \def\VV{\mathbb{V}} \def\WW{\mathbb{W}} \def\XX{\mathbb{X}}
\def\YY{\mathbb{Y}} \def\ZZ{\mathbb{Z}}

\def\fa{\mathfrak{a}} \def\fb{\mathfrak{b}} \def\fc{\mathfrak{c}}
\def\fd{\mathfrak{d}} \def\fe{\mathfrak{e}} \def\ff{\mathfrak{f}}
\def\fg{\mathfrak{g}} \def\fh{\mathfrak{h}} \def\fj{\mathfrak{j}}
\def\fk{\mathfrak{k}} \def\fl{\mathfrak{l}} \def\fm{\mathfrak{m}}
\def\fn{\mathfrak{n}} \def\fo{\mathfrak{o}} \def\fp{\mathfrak{p}}
\def\fq{\mathfrak{q}} \def\fr{\mathfrak{r}} \def\fs{\mathfrak{s}}
\def\ft{\mathfrak{t}} \def\fu{\mathfrak{u}} \def\fv{\mathfrak{v}}
\def\fw{\mathfrak{w}} \def\fx{\mathfrak{x}} \def\fy{\mathfrak{y}}
\def\fz{\mathfrak{z}} \def\fgl{\mathfrak{gl}}  \def\fsl{\mathfrak{sl}}
\def\fso{\mathfrak{so}}  \def\fsp{\mathfrak{sp}}  \def\GL{\mathrm{GL}}
\def\SL{\mathrm{SL}}  \def\SP{\mathrm{SL}}\def\OG{\mathrm{O}}

\def\aa{\mathbf{a}} \def\bb{\mathbf{b}} \def\cc{\mathbf{c}} \def\dd{\mathbf{d}}
\def\ee{\mathbf{e}} \def\ff{\mathbf{f}}
%\def\gg{\mathbf{g}}
\def\hh{\mathbf{h}} \def\ii{\mathbf{i}} \def\jj{\mathbf{j}} \def\kk{\mathbf{k}}
%\def\ll{\mathbf{l}}
\def\mm{\mathbf{m}} \def\nn{\mathbf{n}} \def\oo{\mathbf{o}} \def\pp{\mathbf{p}}
\def\qq{\mathbf{q}} \def\rr{\mathbf{r}} \def\ss{\mathbf{s}} \def\tt{\mathbf{t}}
\def\uu{\mathbf{u}} \def\vv{\mathbf{v}} \def\ww{\mathbf{w}} \def\xx{\mathbf{x}}
\def\yy{\mathbf{y}} \def\zz{\mathbf{z}} \def\zzero{\mathbf{0}}

\def\<{\langle} \def\>{\rangle}
\def\Aut{\mathrm{Aut}}
\def\ch{ \stackrel}
\def\col{\mathrm{col}}
\def\dim{\mathrm{dim}}
\def\End{\mathrm{End}}
\def\ev{\mathrm{ev}}
\def\f{\varphi}
\def\gcd{\mathrm{gcd}}
\def\half{\hbox{$\frac12$}}
\def\Hom{\mathrm{Hom}}
\def\img{\mathrm{img}}
\def\id{\mathrm{id}}
\def\Inn{\mathrm{Inn}}
\def\lcm{\mathrm{lcm}}
\def\normeq{\trianglelefteq}
%\def\ch{ \stackrel{\mathrm{ch}}{\trianglelefteq}}
%\def\normeq{\trianglelefteq}
\def\nul{\mathrm{nullity}}
\def\row{\mathrm{row}}
\def\rk{\mathrm{rank}}
\def\sgn{\mathrm{sgn}}
\def\sp{\mathrm{span}}
\def\supp{\mathrm{supp}}
\def\Syl{\mathrm{Syl}}
\def\tr{\mathrm{tr}}
\def\vep{\varepsilon}

\usepackage{mathabx}
\def\acts{\lefttorightarrow} %group action

%\usepackage{mathtools}
\usepackage{wasysym}

\def\ol{\overline}
\newcommand{\Mod}[1]{\ (\mathrm{mod}\ #1)}

\def\Hfill{$ $\hfill}

% Arrows:
\newcommand\xdhrightarrow[2][]{%
  \mathrel{\ooalign{$\xrightarrow[#1\mkern4mu]{#2\mkern4mu}$\cr%
  \hidewidth$\rightarrow\mkern4mu$}}
}
%\newcommand\dhrightarrow{%
% \mathrel{\ooalign{$\rightarrow$\cr%
% $\mkern3.5mu\rightarrow$}}
%}
\def\dhrightarrow{\twoheadrightarrow}
\def\dhleftarrow{\twoheadleftarrow}
\def\tab{\hspace{10pt}}

% Arrays:
\newcommand\Pmatrix[1]{
  \begin{pmatrix}#1
\end{pmatrix}}
\newcommand\smatrix[1]{\text{\small$
    \begin{pmatrix}#1
\end{pmatrix}$}}
\newcommand\fmatrix[1]{\text{\footnotesize$
    \begin{pmatrix}#1
\end{pmatrix}$}}
\newcommand\tmatrix[1]{\text{\tiny$
    \begin{pmatrix}#1
\end{pmatrix}$}}

\makeatletter
\newcommand{\chareq}{%
  \mathrel{\mathpalette\chRAW\relax}%
}

\newcommand{\chRAW}[2]{%
  \sbox\z@{$#1\LHD$}%
  \sbox\tw@{$#1\leqslant$}%
  \dimen@=\ht\tw@
  \advance\dimen@-\ht\z@
  \advance\dimen@ .3pt
  \ifx#1\displaystyle
  \advance\dimen@ .2pt
  \else
  \ifx#1\textstyle
  \advance\dimen@ .2pt
  \fi
  \fi
  \ooalign{\raisebox{\dimen@}{$\m@th#1\LHD$}\cr$\m@th#1\leqslant$\cr}%
}
\makeatother

%%%%%%%%%%%%%%%%%%%%%%%%%%%%%%
%%%%%%%%%%%%%%%%%%%%%%%%%%%%%%

\def\HW{7}
\def\DUE{April 8th, 2025 by 10:30am}

\title[Homework \HW]{Homework \HW \\
  Math 332, \S S01/02\\
\small Due: \DUE}
\author{}
%\date{}   % Activate to display a given date or no date

\begin{document}
% \maketitle %%% COMMENT THIS OUT and UNCOMMENT the following to give
% yourself a good assignment header:
\begin{flushleft}
  Bram Schuijff\\
  CSCI 387, \S S01/02\\
  Homework \HW\\
  \DUE
\end{flushleft}

\begin{enumerate}
  \item[1.] In order to formulate this problem as a language, we can phrase it
    in the following way:
    \begin{align*}
      L =~&\{\langle G, K\rangle~|~\text{For $G = (V, E)$ a graph with
        $K\subseteq V$,}\\
        &\text{there exists a set of vertices $M\subseteq V\setminus K$
        such that for each $k\in K$,}\\
        &\text{$k$ is adjacent to exactly $n(k)$ vertices in $M$ where
      $n(k)$ is the label of $k$}\}
    \end{align*}
    To demonstrate that $L$ is $\NN\PP$-complete, we must first show that $L$
    is in $\NN\PP$. To do so, we can construct a polynomial-time verifier $D$
    for $L$. When given an input $\langle G, K, w\rangle$, $D$ can check first
    that $K, M\subseteq V$ and furthermore that $K\cap M = \emptyset$.
    Traversing the vertex set of $M$ and $K$ is certainly polynomial time and
    then comparing them against each other is $O(|V|^2)$ time. Therefore, this
    step can be done in polynomial time. To verify that an assignment is valid,
    we can simply iterate through each $k\in K$, check its adjacency list, and
    then see if there are exactly $n(k)$ vertices adjacent to $k$ that are in
    $M$. If there are not, reject. If we have not rejected by the time we have
    iterated through each $k$ in $K$, accept. This can be done in $O(|V|^2)$
    time because that is the most amount of time it can take to check all of
    the adjacent vertices of all vertices in $G$. Therefore, $D$ is polynomial
    time in the size of $\langle G, K\rangle$.

    To check that $D$ verifies $L$, we see first that any witness that $D$ will
    accept necessarily has $M$ and $K$ being disjoint. Any valid witness will
    have this property because if there exists an $e\in M\cap K$, then a mine
    would have been revealed and the game would be over. Then, we simply check
    that every revealed node has exactly as many mines adjacent to it as its
    value says it does. If it does not, then the graph is necessarily
    inconsistent because there exists a revealed node whose value is more or
    less than the number of mines it is adjacent to and we reject. If all
    revealed nodes have precisely the number of mines adjacent to them as their
    s value indicates, then the graph is consistent by definition. Therefore,
    we accept. Thus, $D$ decides $L$ and therefore $L$ is in $\NN\PP$.

    % If our binary expression has a satisfying assignment -> there exists a
    % consistent Minesweeper assignment. Given a binary expression X, turn it
    % into a graph and a revealed set K such that solving the graph and the
    % revealed set would also solve the binary expression

    % Variable gadget: it cannot be that both X and not X are True. Clause
    % gadget: at least one variable from each clause must be True. What is our
    % truth value here? What are we taking to be True? Clearly our variables
    % correspond to the vertices of G but is placing a node into the mine set
    % considered setting that variable as True? Is having one of X and not X
    % being in the mine set and the other being in the revealed set a way to
    % encode our variable gadget?
    To demonstrate that $L$ is $\NN\PP$-complete, we can perform a reduction
    from 3SAT. Given a 3-cnf boolean expression $X$, we are attempting
    to produce a graph and revealed vertex set $G, K$ such that finding whether
    or not $G, K$ is in $L$ would prove the satisfiability of $X$.
\end{enumerate}

\end{document}
