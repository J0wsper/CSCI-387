\documentclass[11pt, reqno]{amsart}
\usepackage[margin=1in]{geometry}
\geometry{letterpaper}
\usepackage[parfill]{parskip} % Activate to begin paragraphs with an
% empty line rather than an indent
\usepackage{amsfonts, amscd, amssymb, amsthm, amsmath}
\usepackage{mathtools} %xmapsto etc
\usepackage{pdfsync} %leaves makers for tex searching
\usepackage{enumerate}

%%% Color %%%---------------------------------------------------------
\usepackage{color}
\usepackage[dvipsnames]{xcolor}
\definecolor{dred}{HTML}{C30101}
\definecolor{dorange}{rgb}{.9,.3,0}
\definecolor{dgrey}{rgb}{.4,.4,.4}
\definecolor{plumb}{HTML}{8105C1}
\definecolor{pumpkin}{HTML}{E47604}
\definecolor{rose}{HTML}{C10091}
\definecolor{dgreen}{HTML}{25A75B}
\definecolor{dblue}{HTML}{0066FF}
\definecolor{cornflower}{HTML}{3256C3}
\definecolor{viridian}{HTML}{099A97}
\definecolor{alert}{HTML}{3256C3}

\newcommand\plumb[1]{{\color{plumb}#1}}
\newcommand\cornfl[1]{{\color{cornflower}#1}}
\newcommand\dgreen[1]{{\color{dgreen}#1}}
\newcommand\viridian[1]{{\color{viridian}#1}}
\newcommand\dblue[1]{{\color{dblue}#1}}
\newcommand\dred[1]{{\color{dred}#1}}
\newcommand\gray[1]{{\color{black!40}#1}}
\newcommand\black[1]{{\color{black}#1}}
\newcommand\pumpk[1]{{\color{pumpkin}#1}}
\newcommand\rose[1]{{\color{rose}#1}}

\newcommand{\NOTE}[1]{{\color{cornflower}#1}}
\newcommand{\alert}[1]{{\color{alert}#1}}
\newcommand{\Alert}[1]{\emph{\color{alert}#1}}

\usepackage{hyperref} %[pdftex,bookmarks]
\hypersetup{
  colorlinks=true,
  linkcolor=viridian,
  filecolor=viridian,
  citecolor=viridian,
  urlcolor=viridian,
  pdfpagemode=FullScreen,
}

%%% Theorems %%%---------------------------------------------------------
\theoremstyle{plain}
\newtheorem*{thm}{Theorem}
\newtheorem*{lemma}{Lemma}
\newtheorem*{prop}{Proposition}
\newtheorem*{cor}{Corollary}
\theoremstyle{definition}
\newtheorem*{defn}{Definition}
\newtheorem*{remark}{Remark}

%%% Environments %%%---------------------------------------------------------
\newenvironment{pf}{\color{black}\medskip
\paragraph*{\emph{Proof}.}}{\hfill \qedsymbol \medskip }
\newenvironment{ans}{\medskip \color{black}
\paragraph*{\emph{Answer}.}}{\hfill \break $~\!\!$ \dotfill \medskip }
\newenvironment{sketch}{\medskip \paragraph*{\emph{Proof sketch}.}}{ \medskip }
\newenvironment{summary}{\medskip \paragraph*{\emph{Summary}.}}{
\hfill \break \rule{1.5cm}{0.4pt} \medskip }
\newcommand\Ans[1]{$ $\hfill {\color{black}\emph{Answer:} {#1}}}
\newcommand{\Hint}[1]{{\small [\emph{Hint:} {#1}]}}

%%% Pictures %%%---------------------------------------------------------
%%% If you need to draw pictures, tikzpicture is one good option.
% Here are some basic things I always use:
\usepackage{tikz}
\usetikzlibrary{scopes}
\usetikzlibrary{automata}
\usetikzlibrary{positioning}
\usepgflibrary{shapes}
\tikzstyle{V}=[draw, fill =black, circle, inner sep=0pt, minimum size=2pt]
\tikzstyle{bV}=[draw, fill =black, circle, inner sep=0pt, minimum size=4pt]
\tikzstyle{over}=[draw=white,double=black,line width=3pt]
\tikzstyle{C}=[draw, thick, fill =white, circle, inner sep=0pt,
minimum size=6pt]

\newcommand\TikZ[1]{
  \begin{matrix}
    \begin{tikzpicture}#1
    \end{tikzpicture}
\end{matrix}}

\newcounter{r}
\newcommand\Part[1]{
  \setcounter{r}{1}
  \foreach \x in {#1}{
    {\ifnum\value{r}=1
      \draw (0,\value{r}-1)--(\x,\value{r}-1);
    \fi}
    \draw (0,\value{r}) to (\x,\value{r});
    \foreach \y in {0, ..., \x} {\draw (\y,\value{r})--(\y,\value{r}-1);}
    \addtocounter{r}{1}
}}
\def\PartUNIT{.175}
%Self-contained tikz images for \Part above.
\newcommand{\PART}[1]{
  \begin{matrix}
    \begin{tikzpicture}[xscale=\PartUNIT, yscale=-\PartUNIT]
      \Part{#1}
    \end{tikzpicture}
  \end{matrix}
}

\def\FOUR{4}\def\ONE{1}\def\FIVE{5}\def\EIGHT{8}\def\THREE{3}\def\TWO{2}\def\SIX{6}\def\SEVEN{7}

%%% Alphabets %%%---------------------------------------------------------
%%% Some shortcuts for my commonly used special alphabets and characters.
\def\cA{\mathcal{A}}\def\cB{\mathcal{B}}\def\cC{\mathcal{C}}\def\cD{\mathcal{D}}\def\cE{\mathcal{E}}\def\cF{\mathcal{F}}\def\cG{\mathcal{G}}\def\cH{\mathcal{H}}\def\cI{\mathcal{I}}\def\cJ{\mathcal{J}}\def\cK{\mathcal{K}}\def\cL{\mathcal{L}}\def\cM{\mathcal{M}}\def\cN{\mathcal{N}}\def\cO{\mathcal{O}}\def\cP{\mathcal{P}}\def\cQ{\mathcal{Q}}\def\cR{\mathcal{R}}\def\cS{\mathcal{S}}\def\cT{\mathcal{T}}\def\cU{\mathcal{U}}\def\cV{\mathcal{V}}\def\cW{\mathcal{W}}\def\cX{\mathcal{X}}\def\cY{\mathcal{Y}}\def\cZ{\mathcal{Z}}

\def\AA{\mathbb{A}} \def\BB{\mathbb{B}} \def\CC{\mathbb{C}} \def\DD{\mathbb{D}}
\def\EE{\mathbb{E}} \def\FF{\mathbb{F}} \def\GG{\mathbb{G}} \def\HH{\mathbb{H}}
\def\II{\mathbb{I}} \def\JJ{\mathbb{J}} \def\KK{\mathbb{K}} \def\LL{\mathbb{L}}
\def\MM{\mathbb{M}} \def\NN{\mathbb{N}} \def\OO{\mathbb{O}} \def\PP{\mathbb{P}}
\def\QQ{\mathbb{Q}} \def\RR{\mathbb{R}} \def\SS{\mathbb{S}} \def\TT{\mathbb{T}}
\def\UU{\mathbb{U}} \def\VV{\mathbb{V}} \def\WW{\mathbb{W}} \def\XX{\mathbb{X}}
\def\YY{\mathbb{Y}} \def\ZZ{\mathbb{Z}}

\def\fa{\mathfrak{a}} \def\fb{\mathfrak{b}} \def\fc{\mathfrak{c}}
\def\fd{\mathfrak{d}} \def\fe{\mathfrak{e}} \def\ff{\mathfrak{f}}
\def\fg{\mathfrak{g}} \def\fh{\mathfrak{h}} \def\fj{\mathfrak{j}}
\def\fk{\mathfrak{k}} \def\fl{\mathfrak{l}} \def\fm{\mathfrak{m}}
\def\fn{\mathfrak{n}} \def\fo{\mathfrak{o}} \def\fp{\mathfrak{p}}
\def\fq{\mathfrak{q}} \def\fr{\mathfrak{r}} \def\fs{\mathfrak{s}}
\def\ft{\mathfrak{t}} \def\fu{\mathfrak{u}} \def\fv{\mathfrak{v}}
\def\fw{\mathfrak{w}} \def\fx{\mathfrak{x}} \def\fy{\mathfrak{y}}
\def\fz{\mathfrak{z}} \def\fgl{\mathfrak{gl}}  \def\fsl{\mathfrak{sl}}
\def\fso{\mathfrak{so}}  \def\fsp{\mathfrak{sp}}  \def\GL{\mathrm{GL}}
\def\SL{\mathrm{SL}}  \def\SP{\mathrm{SL}}\def\OG{\mathrm{O}}

\def\aa{\mathbf{a}} \def\bb{\mathbf{b}} \def\cc{\mathbf{c}} \def\dd{\mathbf{d}}
\def\ee{\mathbf{e}} \def\ff{\mathbf{f}}
%\def\gg{\mathbf{g}}
\def\hh{\mathbf{h}} \def\ii{\mathbf{i}} \def\jj{\mathbf{j}} \def\kk{\mathbf{k}}
%\def\ll{\mathbf{l}}
\def\mm{\mathbf{m}} \def\nn{\mathbf{n}} \def\oo{\mathbf{o}} \def\pp{\mathbf{p}}
\def\qq{\mathbf{q}} \def\rr{\mathbf{r}} \def\ss{\mathbf{s}} \def\tt{\mathbf{t}}
\def\uu{\mathbf{u}} \def\vv{\mathbf{v}} \def\ww{\mathbf{w}} \def\xx{\mathbf{x}}
\def\yy{\mathbf{y}} \def\zz{\mathbf{z}} \def\zzero{\mathbf{0}}

\def\<{\langle} \def\>{\rangle}
\def\Aut{\mathrm{Aut}}
\def\ch{ \stackrel}
\def\col{\mathrm{col}}
\def\dim{\mathrm{dim}}
\def\End{\mathrm{End}}
\def\ev{\mathrm{ev}}
\def\f{\varphi}
\def\gcd{\mathrm{gcd}}
\def\half{\hbox{$\frac12$}}
\def\Hom{\mathrm{Hom}}
\def\img{\mathrm{img}}
\def\id{\mathrm{id}}
\def\Inn{\mathrm{Inn}}
\def\lcm{\mathrm{lcm}}
\def\normeq{\trianglelefteq}
%\def\ch{ \stackrel{\mathrm{ch}}{\trianglelefteq}}
%\def\normeq{\trianglelefteq}
\def\nul{\mathrm{nullity}}
\def\row{\mathrm{row}}
\def\rk{\mathrm{rank}}
\def\sgn{\mathrm{sgn}}
\def\sp{\mathrm{span}}
\def\supp{\mathrm{supp}}
\def\Syl{\mathrm{Syl}}
\def\tr{\mathrm{tr}}
\def\vep{\varepsilon}

\usepackage{mathabx}
\def\acts{\lefttorightarrow} %group action

%\usepackage{mathtools}
\usepackage{wasysym}

\def\ol{\overline}
\newcommand{\Mod}[1]{\ (\mathrm{mod}\ #1)}

\def\Hfill{$ $\hfill}

% Arrows:
\newcommand\xdhrightarrow[2][]{%
  \mathrel{\ooalign{$\xrightarrow[#1\mkern4mu]{#2\mkern4mu}$\cr%
  \hidewidth$\rightarrow\mkern4mu$}}
}
%\newcommand\dhrightarrow{%
% \mathrel{\ooalign{$\rightarrow$\cr%
% $\mkern3.5mu\rightarrow$}}
%}
\def\dhrightarrow{\twoheadrightarrow}
\def\dhleftarrow{\twoheadleftarrow}
\def\tab{\hspace{10pt}}

% Arrays:
\newcommand\Pmatrix[1]{
  \begin{pmatrix}#1
\end{pmatrix}}
\newcommand\smatrix[1]{\text{\small$
    \begin{pmatrix}#1
\end{pmatrix}$}}
\newcommand\fmatrix[1]{\text{\footnotesize$
    \begin{pmatrix}#1
\end{pmatrix}$}}
\newcommand\tmatrix[1]{\text{\tiny$
    \begin{pmatrix}#1
\end{pmatrix}$}}

\makeatletter
\newcommand{\chareq}{%
  \mathrel{\mathpalette\chRAW\relax}%
}

\newcommand{\chRAW}[2]{%
  \sbox\z@{$#1\LHD$}%
  \sbox\tw@{$#1\leqslant$}%
  \dimen@=\ht\tw@
  \advance\dimen@-\ht\z@
  \advance\dimen@ .3pt
  \ifx#1\displaystyle
  \advance\dimen@ .2pt
  \else
  \ifx#1\textstyle
  \advance\dimen@ .2pt
  \fi
  \fi
  \ooalign{\raisebox{\dimen@}{$\m@th#1\LHD$}\cr$\m@th#1\leqslant$\cr}%
}
\makeatother

%%%%%%%%%%%%%%%%%%%%%%%%%%%%%%
%%%%%%%%%%%%%%%%%%%%%%%%%%%%%%

\def\HW{8}
\def\DUE{April 22nd, 2025 by 10:30am}

\title[Homework \HW]{Homework \HW \\
  Math 332, \S S01/02\\
\small Due: \DUE}
\author{}

\begin{document}
\begin{flushleft}
  Bram Schuijff\\
  CSCI 387, \S S01/02\\
  Homework \HW\\
  \DUE
\end{flushleft}

\begin{enumerate}
  \item[1.] Let $A$ be a language that is PSPACE-hard. This means that for any
    $B$ in PSPACE, it may be reduced in polynomial time to $A$. Then, we want
    to show that for any $C$ in NP, $C$ can be reduced in polynomial time to
    $A$. Let $C$ be in NP.
  \item[2.] To demonstrate that $A$ is in L, we can produce a logarithmic space
    decider for $A$. We define such a decider $D$ as follows:
    \begin{enumerate}[1.]
      \item On input $\langle w\rangle$:
      \item\tab Initialize a counter that is set to $0$.
      \item\tab While scanning from left to right across $w$:
      \item\tab\tab If we encounter a left parenthesis, increment the counter
        by $1$.
      \item\tab\tab Else if we encounter a right parenthesis:
      \item\tab\tab\tab If the counter is at $0$, reject.
      \item\tab\tab\tab Else, decrement the counter by $1$.
      \item\tab If the counter is not equal to $0$ by the time we read all of
        $w$, reject. Otherwise, accept.
    \end{enumerate}
    I claim that $D$ decides $A$. This is because if $w$ is a string of
    properly-nested parentheses, then for each left parenthesis, we increment
    our counter by $1$ and for each right parenthesis, we subsequently
    decrement our counter by $1$. Therefore, if our counter is greater than $0$
    by the end of $w$, there is an unclosed left parenthesis and we reject.
    Meanwhile, if our counter is at $0$ and we encounter a right parenthesis,
    this means that right parenthesis will remain unclosed no matter what comes
    after it in $w$ and so we reject. Meanwhile, if our counter is equal to $0$
    by the end of $w$ and we have not rejected, all left parentheses are closed
    and there are no outstanding right parentheses. In this case, we have
    properly-nested parentheses and we accept. Therefore, $D$ decides $A$.

    To demonstrate that $D$ takes log space, we notice that $D$ simply
    maintains a counter. We have discussed previously that counters that count
    occurrences of symbols in the input tape is logarithmic space with regards
    to the input. Therefore, $D$ is log space. Therefore, because $D$ decides
    $A$ and $D$ is logarithmic space, $A$ is in L.
  \item[3.] To show that $A$ is in L, we can produce a logarithmic space
    decider for $A$. We define such a decider $D$ as follows:
    \begin{enumerate}[1.]
      \item On input $x\#w$:
      \item\tab Maintain $4$ counters $l_x,l_w,c_1, c_2$ each initially set to
        $0$.
      \item\tab For each character in $x$:
      \item\tab\tab Increment $l_x$ by $1$.
      \item\tab For each character in $w$:
      \item\tab\tab Increment $l_w$ by $1$.
      \item\tab If $l_x > l_w$, reject.
      \item\tab While $c_1 < l_w - l_x$ + 1:
      \item\tab\tab Scroll back to the $\#$.
      \item\tab\tab While $c_2 < l_x$:
      \item\tab\tab\tab Taking Pythonic string indexing, if $x[c_2]$ is not
        equal to $w[c_1+c_2]$:
      \item\tab\tab\tab\tab Break to the outer loop.
      \item\tab\tab\tab Increment $c_2$ by $1$.
      \item\tab\tab If $c_2$ is equal to $l_x$, accept.
      \item\tab\tab Else, set $c_2 = 0$ and increment $c_1$ by $1$.
      \item\tab Reject.
    \end{enumerate}
    I claim that $D$ decides $A$. This is because for any $x$ and $w$, we start
    by counting up the lengths of $w$ and $x$ and storing them in $l_w$ and
    $l_x$ respectively. Then, if $l_x > l_w$, we reject because it cannot be
    that $x$ is a substring of $w$ if $x$ is longer than $w$. Then, for each
    starting index $c_1$ in $w$, we check whether the $l_x$ characters after
    and including $c_1$ are equal to the characters of $x$ in order. If they
    are, then $w$ contains $x$ as a substring at starting index $c_1$.
    Meanwhile, if we make it to the end of $w$ and have not yet found a
    starting index in $w$ whereby the $l_x$ characters after and including that
    index are equal to $x$, $x$ is not a substring of $w$ and so we reject.
    Therefore, $D$ decides $A$.

    To prove that $D$ is logarithmic space, we find that we maintain a constant
    number of counters which are each bounded by $|x|$ or $|w|$. Because
    maintaining a counter with magnitude in the length of the input string is
    logarithmic with respect to the input string and we have a constant number
    of these counters, we find that $D$ occupies logarithmic space. Therefore,
    because $D$ decides $A$ and $D$ occupies logarithmic space, $A$ is in L.
  \item[4.] To demonstrate that $A_{NFA}$ is NL-complete, we first need to show
    that $A_{NFA}$ is in NL. To do this, we produce a log space verifier $V$
    for $A_{NFA}$ with $x$ as the witness. I will suppose that $x$ contains a
    computation history of $M$ on $w$. This computation history can be
    represented as the current state of $M$ followed by the remaining unread
    tape symbols. As such, $x[i]$ will denote the $i$-th state that $M$ got to
    when run on $w$.
    \begin{enumerate}[1.]
      \item On input $\langle M, w, x\rangle$:
      \item\tab Maintain four counters $l_w, l_x, r_w, p_x$ that are
        initially set to $0$.
      \item\tab For each symbol in $w$:
      \item\tab\tab Increment $l_w$ by $1$.
      \item\tab For each symbol in $x$:
      \item\tab\tab Increment $l_x$ by $1$.
      \item\tab Scroll back to the first symbol on the input tape.
      \item\tab If the first state in $x$ is not the start state, reject.
      \item\tab While $p_x < l_x - 1$:
      \item\tab\tab Initialize a temporary variable $t = 0$.
      \item\tab\tab For each symbol on the computation history's
        input tape at $x[p_x+1]$:
      \item\tab\tab\tab Increment $t$ by $1$.
      \item\tab\tab If $t = r_w$:
      \item\tab\tab\tab If $x[p_x + 1]$ cannot be reached from $x[p_x]$ by an
        $\epsilon$ transition, reject.
      \item\tab\tab Else if $t = r_w - 1$:
      \item\tab\tab\tab If $x[p_x + 1]$ cannot be reached from $x[p_x]$ by
        reading the current symbol, reject.
      \item\tab\tab Clear $t$ from the tape.
      \item\tab If $x[p_x]$ is an accepting state, accept.

    \end{enumerate}
\end{enumerate}

\end{document}
